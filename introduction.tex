\documentclass{report}
\usepackage{helvet}
\renewcommand{\familydefault}{\sfdefault}
\usepackage{color}
\usepackage{geometry}
\geometry{a2paper}
\begin{document}
\fontsize{30pt}{1em}\selectfont
%\color{white}
\begin{itemize}
	\setlength{\itemindent}{0em}
	\item Models of stellar feedback suffer from uncertainty as to how much mass is
	swept up into these hot regions from the surrounding ISM.  As this
	determines both the temperature and the density within feedback-driven
	superbubbles, it therefore determines how long their thermal energy persists
	before being lost to radiative cooling.
	\item Previous studies of wind and supernova driven superbubbles (Chevalier 1974, Mac Low
	\& McCray 1988, etc.) include thermal conduction as a key component.
	\item We show, using the SPH code GASOLINE, that the thermal conduction is the process that determines the mass
	and temperature within a feedback-heated region.
\end{itemize}
\end{document}
