\documentclass[12pt]{report}
%\documentclass[extrafontsizes, 20pt]{memoir}
\pagenumbering{gobble}
\usepackage{helvet}
\renewcommand{\familydefault}{\sfdefault}
\usepackage{color}
\usepackage{geometry}
\geometry{a2paper}
\begin{document}
%\color{white}
\Huge
Stellar feedback is critical to the
evolution of galaxies.  Galaxy simulations without feedback fail to
reproduce observed galaxies on even basic metrics such as star formation rates and rotation
curves.  Recent work has shown that many standard feedback recipes appear to be too weak and lead to galaxies with far too many baryons, primarily in the form of stars (Scannapieco et al. 2012). 

Thacker \& Couchmann (2000) demonstrated that how much  
mass the energy released by stellar feedback is deposited into can significantly
change how rapidly that energy is lost to radiative cooling.  Dalla Vecchia \& Schaye (2012) 
recently implemented a stochastic model that effectively select the temperature of the hot feedback gas by 
tightly controlling mass receiving feedback.  They demonstrate the target temperature selected this dramatically affects galactic outflows.  These studies raise the question of what sets this critical temperature.

Studies of wind and supernova driven superbubbles (Chevalier 1974, Weaver et al. 1977, Mac Low
\& McCray 1988, Silich et al. 1996) consistently include thermal conduction as a key component.  A
schematic diagram of how a star clusters produce a hot feedback bubble is shown in figure 1.
Unlike individual supernovae, superbubbles rapidly form a cold shell and preserve a higher fraction
of the injected energy.  The primary role of conduction is to evaporate the cold outer shell and add mass to the hot bubble.
We have explored this process with extremely high resolution SPH and AMR simulations.

Models of stellar feedback suffer from uncertainty as to how much mass
is swept up into these hot regions from the surrounding ISM.  We have
performed a suite of simulations using the SPH code GASOLINE (Wadsley
et al. 2004) incorporating the effects of the thermal conduction.  In
the following sections we show how conduction regulates the
temperature and hot gas mass in superbubbles associated with young
star clusters.  This, in turn, strongly affects how stellar feedback
acts to regulate star formation and outflows on galactic scales.

\end{document}
