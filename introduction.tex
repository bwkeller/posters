\documentclass{report}
\usepackage{helvet}
\renewcommand{\familydefault}{\sfdefault}
\usepackage{color}
\usepackage{geometry}
\geometry{a2paper}
\begin{document}
\fontsize{30pt}{1em}\selectfont
\color{white}
Modern Simulations have taken a number of approaches to solving the problem of
accurately modelling the effects of stellar feedback.  One of the fundamental
issues with modelling feedback is that of overcooling: a multiphase (hot and
rarified along with cold and dense gas) parcel of gas, while it may have the
same mean density and temperature as a parcel of homogenous, warm gas, should
not radiatively cool as rapidly.  Two main approaches have been taken to prevent
this:  the so-called ``cooling shutoff``, and stochastic feedback.

The key uncertainty in determining both the temperature of the hot bubble (and
thus it's lifetime) as well as the global mixing of enriched SN ejecta is the
small-scale interaction between the three gaseous components involved:  The hot
interior, the dense shell it sweeps up, and the surrounding medium.  As the
density in the shell can be ~10 the density of the hot interior, it rapidly
radiates it's thermal energy, entering the so-called "snowplow phase".  The
evolution of radiating blastwaves has been well-studied (Chevalier 1974, Mac Low
\& McCray 1988, etc.) and have been a basis for a number of feedback models used
more recently (ie Silich 1996).  
\end{document}
