\documentclass[extrafontsizes, 30pt]{memoir}
\pagenumbering{gobble}
\usepackage{helvet}
\renewcommand{\familydefault}{\sfdefault}
\usepackage{color}
\usepackage{geometry}
\geometry{a2paper}
\begin{document}
%\color{white}
Models of stellar feedback suffer from uncertainty as to how much mass is
swept up into these hot regions from the surrounding ISM.  As this
determines both the temperature and the density within feedback-driven
superbubbles, it therefore determines how long their thermal energy persists
before being lost to radiative cooling. 

Dalla Vecchia \& Schaye 2012 propose a stochastic feedback model that relies on
a constant temperature for feedback bubbles.  Mass-loading from conductive
evaporation may constrain the temperatures in physical superbubbles, and
therefore put a limit on models such as this.

Previous studies of wind and supernova driven superbubbles (Chevalier 1974, Mac Low
\& McCray 1988, etc.) include thermal conduction as a key component.

We show, using the SPH code GASOLINE, that the thermal conduction is the process
that determines the mass and temperature within a feedback-heated region.
\end{document}
