\documentclass[extrafontsizes, 30pt]{memoir}
\pagenumbering{gobble}
\usepackage{helvet}
\renewcommand{\familydefault}{\sfdefault}
\usepackage{color}
\usepackage{geometry}
\geometry{a2paper}
\DeclareMathSizes{30}{50}{30}{30}
\begin{document}
%\fontsize{30pt}{1em}\selectfont
%\color{white}
The flow of heat from the hot interior to the shell is
governed by the simple equation:
$$\vec j = \nabla (\kappa T)$$
As the thermal conduction coefficient $\kappa$ strongly depends on the
temperature:
$$\kappa = (6\times10^{-7}  \mathrm{erg s^{-1} cm^{-1} K^{-2/7}})T^{5/2}$$
This heat flux ultimately manifests as a mass flux where the shell evaporates
into the bubble at a rate:
$$\dot M = \frac{4}{25}\frac{\kappa}{k_B} T^{5/2}\frac{4\pi}{R}$$
Thermal conduction acts as a strong self-limiting process on the temperature
of the hot bubble, and ultimately fixes the temperature of the bubble interior.

Numerical simulations of conduction must take care to ensure that timestep
criteria are defined as to integrate these equations stably.
\end{document}
