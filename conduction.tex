\documentclass[extrafontsizes, 30pt]{memoir}
\pagenumbering{gobble}
\usepackage{helvet}
\renewcommand{\familydefault}{\sfdefault}
\usepackage{color}
\usepackage{geometry}
\geometry{a2paper}
\DeclareMathSizes{30}{50}{30}{30}
\begin{document}
%\fontsize{30pt}{1em}\selectfont
\color{white}
A key component of earlier studies was
thermal conduction.  Despite this, contemporary simulations have only considered
thermal conduction in the context of galaxy clusters, and hot IGM.
The flow of heat from the hot interior to the shell is
governed by the simple equation:
$$\vec j = \nabla (\kappa T)$$
As the thermal conduction coefficient $\kappa$ strongly depends on the
temperature:
$$\kappa = \kappa_0 T^{5/2}$$
Thermal conduction can act as a strong self-limiting process on the temperature
of the hot bubble, and ultimately be the process that ``sets'' the temperature
of the bubble interior.
\end{document}
