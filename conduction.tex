\documentclass[12pt]{report}
%\documentclass[extrafontsizes, 20pt]{memoir}
\pagenumbering{gobble}
\usepackage{helvet}
\renewcommand{\familydefault}{\sfdefault}
\usepackage{color}
\usepackage{geometry}
\geometry{a2paper}
\DeclareMathSizes{25}{25}{25}{25}
\begin{document}
%\color{white}
\Huge
The change in energy per unit volume due to thermal conduction by electrons is
governed by the simple equation,
$$\frac{dE}{dt} = \nabla\cdot (\kappa \nabla T)$$.
The thermal conduction coefficient, $\kappa$, strongly depends on the
temperature,
$$\kappa = (6\times10^{-7}  \mathrm{erg s^{-1} cm^{-1} K^{-7/2}})\ T^{5/2}$$.
This heat flux ultimately manifests as a mass flux whereby cold gas evaporates
into the hot gas at a rate (Weaver et al 1977),
$$F_{MASS} = \frac{16 \pi\, \mu \kappa}{25\,k_B}\ \frac{T^{5/2}}{R}$$.
Here $R$ is the radius of the hot bubble.  At high temperatures conduction saturates
with a heat flux given by the electron speed times their density.  
Thermal conduction is a self-limiting process.  This regulates
the temperature of the bubble interior.

Numerical simulations with conduction must take care to ensure that timestep
criteria are defined so as to integrate these equations stably (Jubelgas et al.
2004).  In practice, saturation limits the heat flux due to conduction such that it never requires
timesteps shorter than a fixed fraction of the Courant time.

For the tests shown here we inject energy and mass at a fixed rate so as to compare with established results.  In SPH this was implemented through particle creation at regular intervals at the site of the star cluster.

\end{document}
