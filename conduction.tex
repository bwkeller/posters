\documentclass[extrafontsizes, 30pt]{memoir}
\pagenumbering{gobble}
\usepackage{helvet}
\renewcommand{\familydefault}{\sfdefault}
\usepackage{color}
\usepackage{geometry}
\geometry{a2paper}
\DeclareMathSizes{30}{50}{30}{30}
\begin{document}
%\fontsize{30pt}{1em}\selectfont
\color{white}
A component of early studies of superbubble growth (such as Mac Low \& McCray
1988) was thermal conduction.
The flow of heat from the hot interior to the shell is
governed by the simple equation:
$$\vec j = \nabla (\kappa T)$$
As the thermal conduction coefficient $\kappa$ strongly depends on the
temperature:
$$\kappa = (6*10^{-7}  \mathrm{erg s^{-1} cm^{-1} {-2/7}})T^{5/2}$$
This heat flux ultimately manifests as a mass flux where the shell evaporates
into the bubble at a rate:
$$\dot M = \frac{4}{25}\frac{\kappa}{k_B} T^{5/2}R^3$$
Thermal conduction acts as a strong self-limiting process on the temperature
of the hot bubble, and ultimately fixes the temperature of the bubble interior.
\end{document}
