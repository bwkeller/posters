\documentclass[12pt]{report}
\pagenumbering{gobble}
\usepackage{helvet}
\renewcommand{\familydefault}{\sfdefault}
\usepackage{color}
\usepackage{geometry}
\geometry{a2paper}
\begin{document}
%\color{white}
\Huge
Models of superbubble growth must include conduction in order to
correctly follow their growth.

The amount of mass in hot gas due to stellar feedback is controlled by
thermal conduction and is insensitive to the amount of mass ejected
by the stars.  Omitting conduction can lead to the bubble mass being
underestimated by over an order of magnitude.  

Conduction also regulates the temperature of the bubble.  Simulations without 
conduction severely overestimate the superbubble temperature.  Colder, denser bubbles radiatively cool
more rapidly and are less effective at driving outflows.  

These effects must be incorporated into models of stellar feedback to
accurately model the large scale effects on galaxies.  

\end{document}
