\documentclass{report}
\usepackage{helvet}
\renewcommand{\familydefault}{\sfdefault}
\usepackage{color}
\usepackage{geometry}
\geometry{a2paper}
\begin{document}
\fontsize{30pt}{1em}\selectfont
%\color{white}
\begin{itemize}
	\setlength{\itemindent}{0em}
	\item The mass flux shown in figure 1 results, in the simulation of a
	superbubble shown in figure 2, an additional mass loading due to evaporated shell of
	approximately 6 times the star cluster mass, and more than 20 times the mass of
	the SN ejecta.
	\item This additional mass drops the bubble temperature by more than an
	order of magnitude.  Increased density and decreased temperature means that
	this bubble will radiate away its energy more efficiently than an equivalent
	shell simulated without conduction.
	\item The fragmentation of due to Vishniac instabilites in the thin shell
	shown in figure 5 are artificially supressed in simulations that lack
	thermal conduction,
\end{itemize}
\end{document}
