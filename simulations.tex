\documentclass[extrafontsizes, 30pt]{memoir}
\pagenumbering{gobble}
\usepackage{helvet}
\renewcommand{\familydefault}{\sfdefault}
\usepackage{color}
\usepackage{geometry}
\geometry{a2paper}
\begin{document}
%\color{white}
The mass flux shown in figure 1 results, in the simulation of a
superbubble shown in figure 2, an additional mass loading due to evaporated shell of
approximately 6 times the star cluster mass, and more than 20 times the mass of
the SN ejecta.

This additional mass drops the bubble temperature by more than an
order of magnitude.  Increased density and decreased temperature means that
this bubble will radiate away its energy more efficiently than an equivalent
shell simulated without conduction.

The fragmentation of due to Vishniac instabilites in the thin shell
shown in figure 5 are artificially supressed in simulations that lack
thermal conduction. These are the first SPH simulations to show the growth of
Vishniac instabilities in a feedback-driven superbubble.
\end{document}
