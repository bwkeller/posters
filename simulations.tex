\documentclass[12pt]{report}
%\documentclass[extrafontsizes, 20pt]{memoir}
\pagenumbering{gobble}
\usepackage{helvet}
\renewcommand{\familydefault}{\sfdefault}
\usepackage{color}
\usepackage[margin=0.1in]{geometry}
\geometry{a2paper}
\begin{document}
%\color{white}

\huge
We performed very high resolution simulations of the interface between
hot and cold gas with conduction.  The initial condition is in
pressure equilibrium with a cold medium (1 atom/cc, $10^4$ K) adjacent
to a hot one (0.01 atom/cc, $10^6$ K).  Without conduction nothing
occurs.  The results with conduction are shown in figure 2.  The left
panel clearly shows that the primary effect of conduction is to create
a steady mass flux into the hot medium.  Conduction adds heat to colder
gas as it joins the hot phase so that there is very little net energy
transport out of the hot medium (as is commonly assumed in analytical estimates
such as Weaver et al. 1977).  The thinness of these interfaces means
that the net energy loss, due to cooling, is fairly small.  Thus, in
larger scale simulations, the key effect is adding mass to the hot
bubble.

Figure 3 shows the radial temperature distribution in a superbubble
created by a 30,000 solar mass star cluster with constant energy
output at $3 \times 10^{38}$ erg/s at 30 Myr.  This matches one of the
cases treated in Silich et al (1996).  As shown, conduction reduces
the average bubble temperature to $10^6$ K from $2\times 10^7$ K.
Figure 4 shows the bubble mass (defined by gas exceeding $10^5$ K).
With the addition of conduction, additional mass added by evaporation
closely tracks the mass ejected by stellar feedback with a mass
loading factor of roughly 20.  At 30 Myr the total hot bubble mass is
approximately 6 times the initial star cluster mass.  This mass
loading is in agreement with the results of Silich et
al. (1996), as shown in figure 5.  We have demonstrated that the total bubble mass 
depends only on the energy injected and is insensitive to the mass injected by feedback.

This additional mass decreased the bubble temperature by more than an order of
magnitude (as shown in figure 3).  Increased density and decreased temperature
meant that this bubble radiated away its energy more efficiently than an
equivalent shell simulated without conduction, leading to a smaller bubble radius.

Structure forms in the cold shell due to the Vishniac instability as
shown in figure 6.  These are the first SPH simulations to show the
growth of Vishniac instabilities in a feedback-driven superbubble.
The role of Vishniac instabilities may be sensitive to the state of
the ambient medium (see McLeod \& Whitworth 2013).  Mass loading
through evaporation, in principle, depends on the surface area of the
shell and other factors such as clumpiness (as discussed in Silich et
al. 1996).  However, the self regulating nature of conduction tends to
make the total bubble mass relatively insensitive to these factors.

\end{document}

